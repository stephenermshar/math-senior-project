% \documentclass{beamer}
\documentclass[hyperref={unicode}]{beamer}

\usepackage{amsmath}
\usepackage{amssymb}
\usepackage{array}
\usepackage{booktabs}
\usepackage{cancel}
\usepackage{color}
\usepackage{datetime2}
\usepackage{textcomp}
\usepackage{gensymb}
\usepackage{graphicx}
\usepackage{mathdots}
\usepackage{multirow}
\usepackage[numbers]{natbib}
\usepackage{pgfplots}
\usepackage{siunitx}
\usepackage{tabularx}
\usepackage{tikz}
\usepackage{yhmath}
\usepackage{hyperref}

\DeclareMathOperator{\Ima}{Im}
\DeclareMathOperator{\Ker}{Ker}
\DeclareMathOperator{\nul}{Null}
\newcommand*{\Z}{\mathbb{Z}}

\usetikzlibrary{fadings}
\usetikzlibrary{patterns}
\usetikzlibrary{shadows.blur}
\usetikzlibrary{arrows}
\usetikzlibrary{decorations.markings}
\usetikzlibrary{positioning}

\pgfplotsset{compat=1.16}

\DTMnewdatestyle{mydateformat}{%
  \renewcommand{\DTMdisplaydate}[4]{%
    \number##1\ % year
    \DTMenglishmonthname{##2}\ % Month
    \number##3% day
  }%
  \renewcommand{\DTMDisplaydate}{\DTMdisplaydate}%
}
\DTMsetdatestyle{mydateformat}

\bibliographystyle{amsalpha}

\usefonttheme[onlymath]{serif}
\usetheme{Darmstadt}
\usecolortheme[RGB={80,80,80}]{structure}
\beamertemplatenavigationsymbolsempty

\newcommand{\bigslant}[2]{{\raisebox{.2em}{$#1$}\left/\raisebox{-.2em}{$#2$}\right.}}


\begin{document}




\begin{frame}
    \titlepage
\end{frame}

\begin{frame}{Overview}
	\begin{itemize}[<+->]
		\item Topic Overview
		\item Applications
			\begin{itemize}[<+->]
				\item High Dimensional Data Analysis
				\item Computer Vision
				\item Text Mining
			\end{itemize}
		\item Tools
			\begin{itemize}[<+->]
				\item Rips Complex
				\item Filtrations
			\end{itemize}
		\item Example
	\end{itemize}
\end{frame}

\begin{frame}{What is Persistent Homology?}
	\begin{itemize}[<+->]
		\item ``\textbf{Homology} is a mathematical formalism used to define and identify basic topological features, called holes.'' \cite{wagner}
		\item ``\textbf{Persistent homology} describes the changes in homology when a certain scale parameter is varied.'' \cite{wagner}
	\end{itemize}
\end{frame}

\begin{frame}{Applications}
	\begin{itemize}[<+->]
		\item High Dimensional Data Analysis \cite{ghrist}
	\end{itemize}
\end{frame}


% \begin{frame}{Tools}
% 	\begin{itemize}[<+->]
% 		\item Simplicial Complexes (Rips Complex)
% 		\item Barcodes
% 		\item
% 	\end{itemize}
% \end{frame}

\begin{frame}{References}
	\bibliography{../math496-7-zotero.bib}
\end{frame}




\end{document}
