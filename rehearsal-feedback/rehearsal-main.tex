\documentclass{exam}

% Macro Section
\usepackage{amsmath}
\usepackage{amssymb}
\usepackage{enumitem}
\usepackage{amsthm}
\usepackage{xcolor}
\usepackage{lastpage}
\usepackage{graphicx}

\renewcommand{\bf}[1]{\textbf{#1}}
\renewcommand{\it}[1]{\textit{#1}}
\renewcommand*{\l}{\ensuremath{\left}}
\renewcommand*{\r}{\ensuremath{\right}}

\newcommand*{\setbackgroundcolour}{\pagecolor[rgb]{0.15,0.15,0.15}}
\newcommand*{\settextcolour}{\color[rgb]{0.9,0.9,0.9}}
\newcommand*{\invertbackgroundtext}{\setbackgroundcolour\settextcolour}
\newcommand{\dqt}[1]{``#1''}
\newcommand*{\e}{\ensuremath{\epsilon}}
\newcommand*{\N}{\ensuremath{\mathbb{N}}}
\newcommand*{\R}{\ensuremath{\mathbb{R}}}
\newcommand*{\Z}{\ensuremath{\mathbb{Z}}}
\newcommand*{\Q}{\ensuremath{\mathbb{Q}}}
\newcommand*{\I}{\ensuremath{\mathbb{I}}}
\newcommand*{\contradicts}{\ensuremath{\rightarrow\leftarrow}}
\newcommand{\code}[1]{\texttt{#1}}

\header{Stephen Ermshar}{MATH 497 - Rehearsal Feedback}{2020 April 15, 11:59 pm}
\footer{}{\thepage\ of \pageref{LastPage}}{}
\printanswers
\setlength\parindent{0pt}

\begin{document}

Presentation: Persistent Homology

Advisor: Dr. Foster

\bigskip

I rehearsed my presentation on Tuesday April 14, 2020 at 2pm with Dr. Foster, and then reviewed changes again today (wednesday) at 3pm.

\bigskip

Suggestions:
\begin{itemize}
    \item Correct an error I had where I incorrectly called two groups equal.
    \item Use variables like \(\ell\) or \(\lambda\) to represent elements of interest in groups, like certain cycles that were referenced multiple times, instead of writing out the whole element.
    \item Write groups using generators and angle brackets instead of only writing them as linear combinations.
    \item Avoid referring to an outside video to demonstrate persistence.
    \item Make the connection to the barcodes and persistence diagrams more clear or direct
    \item Don't need to worry too much about focusing on applications or demos at this point.
\end{itemize}

Other Changes Made:
\begin{itemize}
    \item Added a set diagram to explain the kernel and image more clearly on the homology group slide.
    \item Reordered the steps of the homology calculation to make it more streamlined.
    \item Added a demonstration of persistence on a point cloud.
\end{itemize}

\end{document}
