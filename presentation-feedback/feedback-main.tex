\documentclass{exam}

% Macro Section
\usepackage{amsmath}
\usepackage{amssymb}
\usepackage{enumitem}
\usepackage{amsthm}
\usepackage{xcolor}
\usepackage{lastpage}
\usepackage{graphicx}

\renewcommand{\bf}[1]{\textbf{#1}}
\renewcommand{\it}[1]{\textit{#1}}
\renewcommand*{\l}{\ensuremath{\left}}
\renewcommand*{\r}{\ensuremath{\right}}

\newcommand*{\setbackgroundcolour}{\pagecolor[rgb]{0.15,0.15,0.15}}
\newcommand*{\settextcolour}{\color[rgb]{0.9,0.9,0.9}}
\newcommand*{\invertbackgroundtext}{\setbackgroundcolour\settextcolour}
\newcommand{\dqt}[1]{``#1''}
\newcommand*{\e}{\ensuremath{\epsilon}}
\newcommand*{\N}{\ensuremath{\mathbb{N}}}
\newcommand*{\R}{\ensuremath{\mathbb{R}}}
\newcommand*{\Z}{\ensuremath{\mathbb{Z}}}
\newcommand*{\Q}{\ensuremath{\mathbb{Q}}}
\newcommand*{\I}{\ensuremath{\mathbb{I}}}
\newcommand*{\contradicts}{\ensuremath{\rightarrow\leftarrow}}
\newcommand{\code}[1]{\texttt{#1}}

\header{Stephen Ermshar}{MATH 497 - Rehearsal Feedback}{2020 April 29, 5:00 pm}
\footer{}{\thepage\ of \pageref{LastPage}}{}
\printanswers
\setlength\parindent{0pt}

\begin{document}

Presentation: Persistent Homology

Advisor: Dr. Foster

\bigskip

I met with Dr. Foster at 3:00pm on 2020 April 29 to go over the presentation.

\bigskip

Improvements
\begin{itemize}
    \item add examples that can be referred back to throughout the paper.
    \item computer persistent homology for the original homology example.
    \item introduce the 4 point example (or some simple equivalent) at the beginning when talking about connecting points, then return to it when computing homology and persistent homology.
    \item points can be in higher then \(\R^2\) for examples to help illustrate more details.
    \item the same 4 point example can continue to be used when demonstrating the construction of persistence barcodes and diagrams.
\end{itemize}

Expansions:
\begin{itemize}
    \item expand on applications, specifically look for ways to interpret what a hole in data means, or how it can be used to understand the data.
\end{itemize}

\end{document}
