\documentclass[10pt,letterpaper,english,hidelinks]{amsart}

\usepackage{amsmath}
\usepackage{amssymb}
\usepackage{amsthm}
\usepackage{fancyhdr}
\usepackage{graphicx}
\usepackage{pgfplots}
\usepackage{tikz}
\usepackage{datetime2}
\usepackage{natbib}
\usepackage[unicode]{hyperref}
\usepackage{mathtools}
\mathtoolsset{showonlyrefs}

\usetikzlibrary{fadings}
\usetikzlibrary{patterns}
\usetikzlibrary{shadows.blur}
\usetikzlibrary{arrows}
\usetikzlibrary{decorations.markings}
\usetikzlibrary{positioning}
\usetikzlibrary{calc}
\usetikzlibrary{math}

\pgfplotsset{compat=1.16}

\DTMnewdatestyle{mydateformat}{%
  \renewcommand{\DTMdisplaydate}[4]{%
    \number##1\ % year
    \DTMenglishmonthname{##2}\ % Month
    \number##3% day
  }%
  \renewcommand{\DTMDisplaydate}{\DTMdisplaydate}%
}
\DTMsetdatestyle{mydateformat}

% text dimensions
\setlength{\textwidth}{6.5in}
\setlength{\textheight}{9in}

% adjust top margins
\setlength{\topmargin}{0in}
\setlength{\voffset}{-30pt}
\setlength{\headheight}{12pt}

% no room for notes on the side
\setlength{\oddsidemargin}{0in}
\setlength{\evensidemargin}{0in}
\setlength{\marginparwidth}{0in}

% space between paragraphs and lines
\setlength{\parskip}{5pt}
\linespread{1.05} % for final submission
% \linespread{2} % for editing drafts

% Hyphenation
\numberwithin{equation}{section}
\hyphenation{semi-stable}

% Numbering for Theorems, Lemmas, etc.
\theoremstyle{plain}
\newtheorem{theorem}{Theorem}[section]
\newtheorem{corollary}[theorem]{Corollary}
\newtheorem{lemma}[theorem]{Lemma}
\newtheorem{conjecture}[theorem]{Conjecture}
\theoremstyle{definition}
\newtheorem{definition}[theorem]{Definition}
\newtheorem{example}[theorem]{Example}
\newtheorem{remark}[theorem]{Remark}
\numberwithin{equation}{section}

% Header Definitions
\fancyhead[L]{\nouppercase{\rightmark}}
\fancyhead[R]{\nouppercase{\leftmark}}

\pagestyle{fancy}

\newcommand{\cech}{\v{C}ech }
\DeclareMathOperator{\Cech}{\textrm{\v{C}ech}}
\DeclareMathOperator{\Ima}{Im}
\DeclareMathOperator{\Ker}{Ker}
\DeclareMathOperator{\nul}{Null}
\DeclareMathOperator{\boundary}{\partial}
\newcommand*{\Z}{\mathbb{Z}}
\newcommand{\bigslant}[2]{%
  \mathchoice
  {{\raisebox{.2em}{$#1$}\left/\raisebox{-.2em}{$#2$}\right.}}
  {{\raisebox{0em}{$#1$}\left/\raisebox{0em}{$#2$}\right.}}
  {{\raisebox{0em}{$#1$}\left/\raisebox{0em}{$#2$}\right.}}
  {{\raisebox{0em}{$#1$}\left/\raisebox{0em}{$#2$}\right.}}
}
\newcommand{\cross}{\times}
\newcommand{\normalsubgroup}{\triangleleft}
\newcommand{\figref}[1]{[Fig. \ref{#1}]}
\newcommand{\subfigref}[2]{[Fig. \ref{#1}, #2]}
\newcommand{\red}[1]{{\color{red} #1}}
\newcommand{\lightgray}[1]{{\color{lightgray} #1}}


\title{Persistent Homology: Computations and Applications}
\author{Stephen Ermshar}
% \date{\DTMDisplaydate{2020}{5}{15}{}} % first draft due

\begin{document}

\begin{abstract}
    300 word abstract.
\end{abstract}
\maketitle

\section{A Section Title}

Given a collection of points in \(n\)-dimensional space, we want to reveal qualitative properties of an underlying shape that the points may have been sampled from.

\begin{figure}[h!]
    \centering
    \tikzset{every picture/.style={line width=0.75pt}} %set default line width to 0.75pt

\begin{tikzpicture}[x=0.75pt,y=0.75pt,yscale=-0.5,xscale=0.5]
    \begin{scope}[shift={(-200,0)}]
        \begin{axis}[
            axis line style={draw=none},
            tick style={draw=none},
            yticklabels={,,},
            xticklabels={,,}
        ]
        \addplot table [scatter, only marks, x=x, y=y, col sep=comma] {tikz/annulus-point-cloud-data.csv};
        \end{axis}
    \end{scope}

    \draw[->] (100, 100) -- (150, 100);

    \begin{scope}[shift={(-120,-40)}]
        \draw  [draw opacity=0][fill={rgb, 255:red, 74; green, 144; blue, 226 }  ,fill opacity=1 ,even odd rule] (357,153.5) .. controls (357,121.19) and (383.19,95) .. (415.5,95) .. controls (447.81,95) and (474,121.19) .. (474,153.5) .. controls (474,185.81) and (447.81,212) .. (415.5,212) .. controls (383.19,212) and (357,185.81) .. (357,153.5)(329,153.5) .. controls (329,105.73) and (367.73,67) .. (415.5,67) .. controls (463.27,67) and (502,105.73) .. (502,153.5) .. controls (502,201.27) and (463.27,240) .. (415.5,240) .. controls (367.73,240) and (329,201.27) .. (329,153.5) ;
    \end{scope}
\end{tikzpicture}

    \caption{Motivation Example}
\end{figure}

High-dimensional data can be extracted from many sources, including images, health-data, and signals.

\begin{definition}\label{cechc}
    Given a set of points \(P\), and a distance \(r\), let \(B_x(r)\) be the ball with radius \(r\) around the point \(x\).
    \begin{align*}
        \textrm{\v{C}ech}(r) := \{ S \subseteq P | \bigcap_{x\in S} B_x(r) \neq \emptyset \}
    \end{align*}
    \cite{wagner}
\end{definition}

\begin{theorem}\label{obvious}
    A theorem.
\end{theorem}

\begin{proof}
    The theorem's proof.
\end{proof}

\newpage

\bibliographystyle{alpha}
\bibliography{../math496-7-zotero.bib}

\end{document}
