\documentclass[10pt,letterpaper,english,hidelinks]{amsart}

\usepackage{amsmath}
\usepackage{amssymb}
\usepackage{amsthm}
\usepackage{fancyhdr}
\usepackage{graphicx}
\usepackage{pgfplots}
\usepackage{tikz}
\usepackage{datetime2}
\usepackage{natbib}
\usepackage[unicode]{hyperref}
\usepackage{mathtools}
\mathtoolsset{showonlyrefs}

\usetikzlibrary{fadings}
\usetikzlibrary{patterns}
\usetikzlibrary{shadows.blur}
\usetikzlibrary{arrows}
\usetikzlibrary{decorations.markings}
\usetikzlibrary{positioning}
\usetikzlibrary{calc}
\usetikzlibrary{math}

\pgfplotsset{compat=1.16}

\DTMnewdatestyle{mydateformat}{%
  \renewcommand{\DTMdisplaydate}[4]{%
    \number##1\ % year
    \DTMenglishmonthname{##2}\ % Month
    \number##3% day
  }%
  \renewcommand{\DTMDisplaydate}{\DTMdisplaydate}%
}
\DTMsetdatestyle{mydateformat}

% text dimensions
\setlength{\textwidth}{6.5in}
\setlength{\textheight}{9in}

% adjust top margins
\setlength{\topmargin}{0in}
\setlength{\voffset}{-30pt}
\setlength{\headheight}{12pt}

% no room for notes on the side
\setlength{\oddsidemargin}{0in}
\setlength{\evensidemargin}{0in}
\setlength{\marginparwidth}{0in}

% space between paragraphs and lines
\setlength{\parskip}{5pt}
\linespread{1.05} % for final submission
% \linespread {2} % for editing drafts

% Hyphenation
\numberwithin{equation}{section}
\hyphenation{semi-stable}

% Numbering for Theorems, Lemmas, etc.
% \theoremstyle{plain}
\theoremstyle{definition} % on recommendation, for consistency.
\newtheorem{theorem}{Theorem}[section]
\newtheorem{corollary}[theorem]{Corollary}
\newtheorem{lemma}[theorem]{Lemma}
\newtheorem{conjecture}[theorem]{Conjecture}
\theoremstyle{definition}
\newtheorem{definition}[theorem]{Definition}
\newtheorem{example}[theorem]{Example}
\newtheorem{remark}[theorem]{Remark}
\numberwithin{equation}{section}

% Header Definitions
\fancyhead[L]{\nouppercase{\rightmark}}
\fancyhead[R]{\nouppercase{\leftmark}}

\pagestyle{fancy}

\newcommand{\cech}{\v{C}ech }
\DeclareMathOperator{\Cech}{\textrm{\v{C}ech}}
\DeclareMathOperator{\Ima}{Im}
\DeclareMathOperator{\Ker}{Ker}
\DeclareMathOperator{\nul}{Null}
\DeclareMathOperator{\boundary}{\partial}
\newcommand*{\Z}{\mathbb{Z}}
\newcommand{\bigslant}[2]{%
  \mathchoice
  {{\raisebox{.2em}{$#1$}\left/\raisebox{-.2em}{$#2$}\right.}}
  {{\raisebox{0em}{$#1$}\left/\raisebox{0em}{$#2$}\right.}}
  {{\raisebox{0em}{$#1$}\left/\raisebox{0em}{$#2$}\right.}}
  {{\raisebox{0em}{$#1$}\left/\raisebox{0em}{$#2$}\right.}}
}
\newcommand{\cross}{\times}
\newcommand{\normalsubgroup}{\triangleleft}
\newcommand{\figref}[1]{[Fig.~\ref{#1}]}
\newcommand{\subfigref}[2]{[Fig.~\ref{#1},~#2]}
\newcommand{\red}[1]{{\color{red} #1}}
\newcommand{\lightgray}[1]{{\color{lightgray} #1}}
